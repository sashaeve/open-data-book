\chapter{Ресурси}
\label{sec:references}

\begin{enumerate}
    \item \textbf{visualizing.org} \\
    \url{http://www.visualizing.org} \\
    Спільнота і інформаційна платформа про візуалізації даних.

    \item \textbf{Google Chart Tools} \\
    \url{http://code.google.com/apis/chart/} \\
    Javascript API від Google для простого створення таблиць візуалізації для постійно змінюються даних.

    \item \textbf{GeoCommons} \\
    \url{http://geocommons.com} \\
    Інструментарій, спільнота і сервіс візуалізації для спільного використання геоданих.

    \item \textbf{Quadrigram} \\
    \url{http://www.quadrigram.com} \\
    Професійна платформа з можливістю платного побудови кастомізованих візуалізацій.

    \item \textbf{JavaScript InfoVis Toolkit} \\
    \url{http://thejit.org} \\
    Javascript-інструментарій для створення і підтримки візуалізації різного роду графіків.

    \item \textbf{d3.js - Data-Driven Documents} \\
    \url{http://mbostock.github.com/d3/} \\
    Маленька і дуже гнучка JavaScript бібліотека з відкритим вихідним кодом для управління документами на основі даних (наприклад, для генерації HTML5 або SVG-діаграми з даних).

    \item \textbf{Google Public Data Explorer} \\
    \url{http://www.google.com/publicdata/home} \\
    Каталог загального набору даних та інструмент для публікації і візуалізації великих наборів даних.

    \item \textbf{Maps Marker WP-Plugin} \\
    \url{http://www.mapsmarker.com} \\
    Wordpress-плагін для відображення карти з анотацією видатних пам'яток в блозі Wordpress.

    \item \textbf{DataMaps.eu - map your data} \\
    \url{http://www.datamaps.eu/} \\
    Інструмент для створення привабливих карт візуалізації, які можуть бути створені в браузері через веб-сайт без знання програмування.

    \item \textbf{Ushahidi} \\
    \url{http://www.ushahidi.com} \\
    Відкрите програмне забезпечення для збору, візуалізації та інтерактивного відображення на основі визначення місця розташування даних в реальному часі (наприклад, від надзвичайних ситуацій, політичних виборів і т.д.).

    \item \textbf{Eclipse BIRT} \\
    \url{http://www.eclipse.org/birt/phoenix/} \\
    Система звітності eclipse для створення візуально привабливих звітів великих обсягів даних.

    \item \textbf{Chartle.net} \\
    \url{http://www.chartle.net} \\
    Безкоштовне інтерактивний онлайн-додаток по створенню графіків. Інтуїтивно зрозумілий інтерфейс, не вимагає спеціальних навичок, проте і набір можливостей обмежений. Застосовується, коли потрібен швидкий результат: круглі і стовпчасті діаграми, лінійні графіки, динамічні схеми, географічна карта двох видів. Підсумкова візуалізація інтерактивна, і її код легко вбудовується в html-сторінку.

    \item \textbf{Hohli} \\
    \url{http://charts.hohli.com/} \\
    Простий і скромний інтерактивний безкоштовний онлайн-інструмент для візуалізації даних за допомогою стандартного набору діаграм. (Немає можливості створювати карти.)

    \item \textbf{IBM Many Eyes} \\
    \url{http://www-958.ibm.com/software/data/cognos/manyeyes/} \\
    Популярний онлайн-інструмент для візуалізації даних. Безкоштовний. Є можливість спільної роботи над проектами.

    \item \textbf{TagCrowd} \\
    \url{http://www.tagcrowd.com} \\
    Онлайн-додаток для аналізу і візуалізації частотності вживання слів у тексті. Безкоштовний.

    \item \textbf{Wordle} \\
    \url{http://www.wordle.net} \\
    Онлайн-додаток для аналізу і візуалізації частотності вживання слів у тексті. Безкоштовний.

    \item \textbf{Tableau} \\
    \url{http://www.tableausoftware.com} \\
    Сімейство програм для візуалізації даних. Асортимент опцій для кастомізації, а також можливість комбінувати кілька візуалізацій на одній панелі. За підсумками створення візуалізації можна отримати html-код, який можна вбудувати в веб-сторінку. Графіки інтерактивні. Закритий софт, однак, є безкоштовна версія з великою кількістю доступних можливостей (Tableau Public).

    \item \textbf{Dundas} \\
    \url{http://www.dundas.com} \\
    Програмне забезпечення для створення інтерактивних візуалізацій. Може обробляти великі масиви даних. Створює візуалізації, в числі іншого, у вигляді панелей з декількох компонентів, що дозволяє одночасно представити кілька вимірів. Працює онлайн, комерційне, платне. Пропонують 45-денний безкоштовний випробувальний термін.

    \item \textbf{Leximancer} \\
    \url{https://www.leximancer.com} \\
    Професійна програма для аналізу тексту та візуалізації результатів цього аналізу. Комерційна, платна, кроссплатформенная.

    \item \textbf{SIMILE Widgets} \\
    \url{http://www.simile-widgets.org} \\
    Збірка різноманітних віджетів і їх кодів. Коди відкриті з можливістю адаптувати під свої потреби, але для цього потрібні відповідні навички. Серед іншого, є інструменти, що дозволяють обробляти великі масиви даних і конструювати карти, тайм-лайни, інтерактивні таблиці і багато іншого. Інструмент Exibit дозволяє створювати цілі інтерактивні веб-сторінки з можливістю пошуку і самостійного дослідження представленої бази даних.

    \item \textbf{GeoCommons} \\
    \url{http://geocommons.com} \\
    Безкоштовний веб-інструмент зі створення карт на основі даних.

    \item \textbf{Gephi} \\
    \url{http://gephi.org} \\
    Програмне забезпечення для візуалізації графів. Використовується як один з інструментів аналізу соцмереж. Безкоштовний, відкритий код, кроссплатформлений.

    \item \textbf{Graphviz} \\
    \url{http://www.graphviz.org} \\
    Програма для візуалізації графів. Відкритий код, кросплатформенна, безкоштовна.

    \item \textbf{NewRadial} \\
    \url{http://sourceforge.net/projects/newradial/} \\
    Комплекс інструментів для візуального представлення нечислових даних (в тому числі зображень).

    \item \textbf{Prefuse} \\
    \url{http://www.prefuse.org} \\
    Великий набір різних інструментів для створення складних інтерактивних візуалізацій. Вимагають вміння програмувати. Безкоштовний, код відкритий.

    \item \textbf{ТЕКСТИ.ORG.UA} \\
    \url{http://texty.org.ua/pg/blog/infoviz} \\
    Блог: Інфовіз: інфографіка, інтерактивна візуалізація даних.

    \item \textbf{Charted} \\
    \url{http://www.charted.co/} \\
    Швидка візуалізація CSV файлів.

    \item \textbf{Venngage} \\
    \url{https://venngage.com/} \\
    Комерційний сервіс для створення інфографіки.

    \item \textbf{Dipity} \\
    \url{http://www.dipity.com/} \\
    Створення гарних таймлайнів.

    \item \textbf{Easily} \\
    \url{https://piktochart.com/} \\
    Генератор інфографіки.

    \item \textbf{Automatic Infographic Generator} \\
    \url{http://petercv.com/aig/} \\
    Генератор інфографіки.
\end{enumerate}