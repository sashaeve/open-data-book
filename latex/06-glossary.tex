\chapter{Глосарій}
\label{sec:glossary}

\begin{description}
\item[Абстрактна модель] — модель, що відображає загальні характеристики явища. Вона представляє інформацію про якісні характеристики об'єкта чи явища.
\item[Відкрита ліцензія] — документ, що визначає права й обмеження щодо об'єкта, регламентує вільне поширення контенту та/або програмного забезпечення. Свобода означає можливість використання, аналогічну свободі слова.
\item[Відкриті дані (Open Data)] — систематизована інформація, доступна через Інтернет у форматі, що дозволяє автоматизовану обробку, вільний і безоплатний доступ, а також подальше використання.
\item[Відкриті державні дані (Open Government Data)] — публічна інформація у вигляді відкритих даних про діяльність органів влади або створена в результаті їх діяльності.
\item[Власник інформації] — особа, яка створила інформацію або отримала право дозволяти чи обмежувати доступ до неї на підставі закону чи договору.
\item[Ідентифікатор набору відкритих даних (URN, Uniform Resource Name)] — постійний строковий параметр у визначеному форматі для ідентифікації та формування назв файлів набору, що включає інформацію про орган-власник, систему, тип даних, дати публікації й оновлення тощо.
\item[Інтерфейс прикладного програмування (API, Application Programming Interface)] — набір класів, функцій, структур і констант, доступних як сервіси через Інтернет для використання у прикладних програмах.
\item[Машинозчитувані дані] — дані у форматах, придатних для автоматичного використання.
\item[Метадані (метаінформація)] — структуровані дані, що визначають характеристики відкритих даних для їх ідентифікації та обробки.
\item[Набір відкритих даних (набір даних)] — сукупність однорідних записів, що містять структуровану інформацію — поля даних і метаінформацію.
\end{description}
