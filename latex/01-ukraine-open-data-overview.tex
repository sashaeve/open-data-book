\chapter{Про стан розвитку відкритих даних в Україні}

\section{Запущено єдиний державний портал відкритих даних data.gov.ua}

Перша версія порталу була розроблена волонтерами із організації Socialboost за підтримки міжнародних організацій і компанії Microsoft. На сьогоднішній день портал переданий на баланс Державного агентства з питань електронного урядування. На порталі доступно майже 6000 наборів даних від 700+ розпорядників, хоча якісних наборів даних у відсотковому співвідношенні не дуже багато.

\section{Запущено систему держваних закупівель ProZorro}

\href{https://prozorro.gov.ua}{ProZorro} – електронна система публічних закупівель яка прийшла на зміну паперовим держтендерам.

У Законі України «Про публічні закупівлі» передбачається:

\begin{itemize}
    \item Запровадження обов'язковості проведення процедур через електронну систему. (Перший етап — обов'язковість проведення процедур через електронну систему поширюється на головних розпорядників коштів та монополістів (з 1 квітня 2016 року), на другому (з 1 серпня) – на всіх замовників;
    \item Запровадження електронного аукціону, який передбачає автоматичну оцінку тендерних пропозицій;
    \item Визначення нових понять «авторизований електронний майданчик», «електронна система закупівель», «централізована закупівельна організація», «система хмарних обчислень»;
    \item Замість 5 процедур залишити 3 (відкриті торги, конкурентний діалог, переговорна процедура»;
    \item Зміна термінології, зокрема, замість терміну «державна закупівля» вводиться термін «публічна закупівля»; замість термінів «конкурс», «документація конкурсних торгів», «пропозиція конкурсних торгів», «комітет з конкурсних торгів», вводяться поняття «тендер», «тендерна документація», «тендерна пропозиція», «тендерний комітет».
\end{itemize}

ProZorro отримала міжнародну премію у сфері публічних закупівель Public Sector Procurement Award за створення і впровадження електронної системи з унікальною архітектурою. Розробка цієї системи на базі програмного забезпечення з відкритим кодом була здійснена у партнерстві влади, бізнесу та громадськості та адмініструвалася антикорупційною організацією \href{https://uk.wikipedia.org/wiki/%D0%A2%D1%80%D0%B0%D0%BD%D1%81%D0%BF%D0%B5%D1%80%D0%B5%D0%BD%D1%81%D1%96_%D0%86%D0%BD%D1%82%D0%B5%D1%80%D0%BD%D0%B5%D1%88%D0%BD}{Transparency International Україна}.

З 1 серпня 2016 року ProZorro є обов'язковою системою для всіх державних замовників при закупівлі від 200 тис. грн для товарів і від 1,5 млн грн для робіт.

\section{Запущено офіційний портал публічних фінансів України Є-Data}

Є-Data — це офіційний державний інформаційний портал у мережі Інтернет, на якому оприлюднюється інформація про використання публічних коштів та реалізується ідея «Прозорого бюджету». Задача - забезпечити повну прозорість державних фінансів та задовольнить право громадськості на доступ до інформації.

Основні законодавчі та нормативні засади для створення проекту:

\begin{itemize}
    \item \href{http://zakon3.rada.gov.ua/laws/show/183-19}{Закон України «Про відкритість використання публічних коштів»}
    \item План заходів з виконання Програми діяльності Кабінету Міністрів України та Стратегії сталого розвитку "Україна - 2020" у 2015 році п. 95: «Впровадження інтегрованої інформаційно-аналітичної системи "Прозорий бюджет" з метою забезпечення доступності інформації про державні фінанси для суспільних потреб із забезпеченням відкритої звітності за всіма коштами, використаними отримувачами бюджетних коштів..»
    \item Коаліційна угода Парламенту VIII скликання п. 2.5.10.: «Запровадження системи «Прозорий бюджет» з метою забезпечення доступності інформації про державні фінанси для суспільних потреб.
\end{itemize}

З 15 вересня 2015 року на порталі оприлюднюються всі трансакції Державної казначейської служби, з листопада на ньому доступна інформація про використання коштів державного і місцевих бюджетів, а у січні інформацію почали розкривати суб'єкти господарювання державної і комунальної власності, у статутному капіталі яких державна або комунальна частка акцій (часток, паїв) перевищує 50 відсотків.

До 2018 року планується забезпечити повний функціонал Інтегрованої інформаційно-аналітичної системи «Прозорий бюджет», яка своєю чергою торкнеться змін бюджетних процесів Міністерства фінансів, автоматизації систем ДФС та Казначейства, автоматизації систем обліку та звітності на місцевих рівнях.

\section{Запущено пошуково-аналітичну систему 007}

\href{https://www.facebook.com/pointOOSeven}{Пошуково-аналітична система 007} – це веб-сервіс на основі відкритих даних про використання публічних коштів. Проект передбачає сервіс пошуку та візуалізації даних з відкритих джерел про використання державою бюджетних коштів. Його ідея – дати громадськості інструмент контролю влади та можливість стежити за бюджетними витратами, збирати докази зловживань і швидко переводити боротьбу з корупцією в правове поле. Основний акцент зроблено на простоті використання та представлення специфічної інформації з масивів великих даних. Сайт був відкритий 8 квітня 2016 року.

\section{Нормативно-правове забезпечення}

9 квітня 2015 року Верховна Рада України ухвалила Закон України № 319 «\href{http://zakon3.rada.gov.ua/laws/show/319-19}{Про внесення змін до деяких законів України щодо доступу до публічної інформації у формі відкритих даних}». Зазначеним Законом внесені зміни до Закону України «Про доступ до публічної інформації» з метою визначення базових норма та засад розвитку відкритих даних в Україні, а саме:

\begin{enumerate}
    \item Публічна інформація у формі відкритих даних - це публічна інформація у форматі, що дозволяє її автоматизоване оброблення електронними засобами, вільний та безоплатний доступ до неї, а також її подальше використання;
    \item Розпорядники інформації зобов'язані надавати публічну інформацію у формі відкритих даних на запит, оприлюднювати і регулярно оновлювати її на єдиному державному веб-порталі відкритих даних та на своїх веб-сайтах;
    \item Будь-яка особа може вільно копіювати, публікувати, поширювати, використовувати, у тому числі в комерційних цілях, у поєднанні з іншою інформацією або шляхом включення до складу власного продукту, публічну інформацію у формі відкритих даних з обов'язковим посиланням на джерело отримання такої інформації.
\end{enumerate}

21 жовтня 2015 року затверджено Постановою КМУ № 835 «\href{http://zakon5.rada.gov.ua/laws/show/835-2015-%D0%BF}{Про затвердження Положення про набори даних, які підлягають оприлюдненню у формі відкритих даних}», якою визначено вимоги до формату і структури наборів даних, а також затверджено перелік пріоритетних наборів даних, які підлягають оприлюдненню (більше 300 наборів). Постановою чітко визначений перелік форматів для оприлюднення відкритих даних в залежності від їх виду:

\section{Формати даних для оприлюднення у формі відкритих даних}

\textbf{Текстові дані}  
TXT, RTF, ODT, DOC(X), PDF (с текстовим змістом, скановане зображення), (X)HTML

\textbf{Структуровані дані}  
RDF, XML, JSON, CSV, XLS(X), ODS, YAML

\textbf{Графічні дані}  
GIF, TIFF, JPG (JPEG), PNG

\textbf{Відеодані}  
MPEG, MKV, AVI, FLV, MKS, MK3D

\textbf{Аудіодані}  
MP3, WAV, MKA

\textbf{Дані, розроблені з використанням програми Macromedia Flash}  
SWF, FLV

\textbf{Архіви даних}  
ZIP, 7z, Gzip, Bzip2

Також, постановою передбачено, що державні реєстри, які постійно оновлюються, мають бути відкриті за допомогою API.

З метою забезпечення ефективного функціонування Єдиного державного веб-порталу відкритих даних та підвищення відкритості і прозорості діяльності органів виконавчої влади та місцевого самоврядування, Державним агентством з питань електронного урядування в лютому 2016 року розроблено проект постанови Кабінету Міністрів України «Деякі питання оприлюднення публічної інформації у формі відкритих даних», які знаходяться зараз на етапі обговорення.
