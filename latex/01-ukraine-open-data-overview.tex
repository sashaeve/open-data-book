\chapter{Про стан розвитку відкритих даних в Україні}

\textbf{За останні два роки Україна досягла великого прогресу в сфері відкритих даних, але все ще знаходиться досить далеко в міжнародних рейтингах.}

\subsection{Запущено єдиний державний портал відкритих даних data.gov.ua}

Перша версія порталу була розроблена волонтерами із організації Socialboost за підтримки міжнародних організацій і компанії Microsoft. На сьогоднішній день портал переданий на баланс Державного агентства з питань електронного урядування. На порталі доступно майже 6000 наборів даних від 700+ розпорядників, хоча якісних наборів даних у відсотковому співвідношенні не дуже багато.

\subsection{Запущено систему держваних закупівель ProZorro}

\href{https://prozorro.gov.ua}{ProZorro} – електронна система публічних закупівель яка прийшла на зміну паперовим держтендерам.

У Законі України «Про публічні закупівлі» передбачається:
\begin{itemize}
    \item Запровадження обов'язковості проведення процедур через електронну систему. (Перший етап — обов'язковість проведення процедур через електронну систему поширюється на головних розпорядників коштів та монополістів (з 1 квітня 2016 року), на другому (з 1 серпня) – на всіх замовників;
    \item Запровадження електронного аукціону, який передбачає автоматичну оцінку тендерних пропозицій;
    \item Визначення нових понять «авторизований електронний майданчик», «електронна система закупівель», «централізована закупівельна організація», «система хмарних обчислень»;
    \item Замість 5 процедур залишити 3 (відкриті торги, конкурентний діалог, переговорна процедура»;
    \item Зміна термінології, зокрема, замість терміну «державна закупівля» вводиться термін «публічна закупівля»; замість термінів «конкурс», «документація конкурсних торгів», «пропозиція конкурсних торгів», «комітет з конкурсних торгів», вводяться поняття «тендер», «тендерна документація», «тендерна пропозиція», «тендерний комітет».
\end{itemize}

ProZorro отримала міжнародну премію у сфері публічних закупівель Public Sector Procurement Award за створення і впровадження електронної системи з унікальною архітектурою. Розробка цієї системи на базі програмного забезпечення з відкритим кодом була здійснена у партнерстві влади, бізнесу та громадськості та адмініструвалася антикорупційною організацією \href{https://uk.wikipedia.org/wiki/%D0%A2%D1%80%D0%B0%D0%BD%D1%81%D0%BF%D0%B5%D1%80%D0%B5%D0%BD%D1%81%D1%96_%D0%86%D0%BD%D1%82%D0%B5%D1%80%D0%BD%D0%B5%D1%88%D0%BD}{Transparency International Україна}.

З 1 серпня 2016 року ProZorro є обов'язковою системою для всіх державних замовників при закупівлі від 200 тис. грн для товарів і від 1,5 млн грн для робіт.

\subsection{Запущено офіційний портал публічних фінансів України Є-Data}

Є-Data — це офіційний державний інформаційний портал у мережі Інтернет, на якому оприлюднюється інформація про використання публічних коштів та реалізується ідея «Прозорого бюджету». Задача - забезпечити повну прозорість державних фінансів та задовольнить право громадськості на доступ до інформації.

Основні законодавчі та нормативні засади для створення проекту:
\begin{itemize}
    \item \href{http://zakon3.rada.gov.ua/laws/show/183-19}{Закон України «Про відкритість використання публічних коштів»}
    \item План заходів з виконання Програми діяльності Кабінету Міністрів України та Стратегії сталого розвитку "Україна - 2020" у 2015 році п. 95: «Впровадження інтегрованої інформаційно-аналітичної системи "Прозорий бюджет" з метою забезпечення доступності інформації про державні фінанси для суспільних потреб із забезпеченням відкритої звітності за всіма коштами, використаними отримувачами бюджетних коштів..»
    \item Коаліційна угода Парламенту VIII скликання п. 2.5.10.: «Запровадження системи «Прозорий бюджет» з метою забезпечення доступності інформації про державні фінанси для суспільних потреб.
\end{itemize}

З 15 вересня 2015 року на порталі оприлюднюються всі трансакції Державної казначейської служби, з листопада на ньому доступна інформація про використання коштів державного і місцевих бюджетів, а у січні інформацію почали розкривати суб'єкти господарювання державної і комунальної власності, у статутному капіталі яких державна або комунальна частка акцій (часток, паїв) перевищує 50 відсотків.

До 2018 року планується забезпечити повний функціонал Інтегрованої інформаційно-аналітичної системи «Прозорий бюджет», яка своєю чергою торкнеться змін бюджетних процесів Міністерства фінансів, автоматизації систем ДФС та Казначейства, автоматизації систем обліку та звітності на місцевих рівнях.

\subsection{Запущено пошуково-аналітичну систему 007}

\href{https://www.facebook.com/pointOOSeven}{Пошуково-аналітична система 007} – це веб-сервіс на основі відкритих даних про використання публічних коштів. Проект передбачає сервіс пошуку та візуалізації даних з відкритих джерел про використання державою бюджетних коштів. Його ідея – дати громадськості інструмент контролю влади та можливість стежити за бюджетними витратами, збирати докази зловживань і швидко переводити боротьбу з корупцією в правове поле. Основний акцент зроблено на простоті використання та представлення специфічної інформації з масивів великих даних. Сайт був відкритий 8 квітня 2016 року.

\subsection{Нормативно-правове забезпечення}

9 квітня 2015 року Верховна Рада України ухвалила Закон України № 319 \href{http://zakon3.rada.gov.ua/laws/show/319-19}{«Про внесення змін до деяких законів України щодо доступу до публічної інформації у формі відкритих даних»}. Зазначеним Законом внесені зміни до Закону України «Про доступ до публічної інформації» з метою визначення базових норма та засад розвитку відкритих даних в Україні, а саме:

\begin{enumerate}
    \item Публічна інформація у формі відкритих даних - це публічна інформація у форматі, що дозволяє її автоматизоване оброблення електронними засобами, вільний та безоплатний доступ до неї, а також її подальше використання;
    \item Розпорядники інформації зобов'язані надавати публічну інформацію у формі відкритих даних на запит, оприлюднювати і регулярно оновлювати її на єдиному державному веб-порталі відкритих даних та на своїх веб-сайтах;
    \item Будь-яка особа може вільно копіювати, публікувати, поширювати, використовувати, у тому числі в комерційних цілях, у поєднанні з іншою інформацією або шляхом включення до складу власного продукту, публічну інформацію у формі відкритих даних з обов'язковим посиланням на джерело отримання такої інформації.
\end{enumerate}

21 жовтня 2015 року затверджено Постановою КМУ № 835 \href{http://zakon5.rada.gov.ua/laws/show/835-2015-%D0%BF}{«Про затвердження Положення про набори даних, які підлягають оприлюдненню у формі відкритих даних»}, якою визначено вимоги до формату і структури наборів даних, а також затверджено перелік пріоритетних наборів даних, які підлягають оприлюдненню (більше 300 наборів). Постановою чітко визначений перелік форматів для оприлюднення відкритих даних в залежності від їх виду:

\textbf{Текстові дані} \\
TXT, RTF, ODT, DOC(X), PDF (с текстовим змістом, скановане зображення), (X)HTML
21 жовтня 2015 року затверджено Постановою КМУ № 835 «[Про затвердження Положення про набори даних, які підлягають оприлюдненню у формі відкритих даних](http://zakon5.rada.gov.ua/laws/show/835-2015-%D0%BF)», якою визначено вимоги до формату і структури наборів даних, а також затверджено перелік пріоритетних наборів даних, які підлягають оприлюдненню (більше 300 наборів). Постановою чітко визначений перелік форматів для оприлюднення відкритих даних в залежності від їх виду:

**Текстові дані**  
TXT, RTF, ODT, DOC(X), PDF (с текстовим змістом, скановане зображення), (X)HTML

**Структуровані дані**  
RDF, XML, JSON, CSV, XLS(X), ODS, YAML

**Графічні дані**  
GIF, TIFF, JPG (JPEG), PNG

**Видеодані**  
MPEG, MKV, AVI, FLV, MKS, MK3D

**Аудіодані**  
MP3, WAV, MKA

**Дані, розроблені з використанням програми Macromedia Flash**  
SWF, FLV

**Архіви даних**  
ZIP, 7z, Gzip, Bzip2

Також, постановою передбачено, що державні реєстри, які постійно оновлюються, мають бути відкриті за допомогою API.

З метою забезпечення ефективного функціонування Єдиного державного веб-порталу відкритих даних та підвищення відкритості і прозорості діяльності органів виконавчої влади та місцевого самоврядування, Державним агентством з питань електронного урядування в лютому 2016 року розроблено проект постанови Кабінету Міністрів України «Деякі питання оприлюднення публічної інформації у формі відкритих даних», які знаходяться зараз на етапі обговорення. 

## Оцінка готовності України до розвитку відкритих даних

З метою визначення поточного стану розвитку та готовності України до розвитку відкритих даних, а також планування подальший дій, проведено оцінку готовності України до розвитку відкритих даних за методикою Всесвітнього банку ODRA.

Оцінка проводилась за восьма напрямами:

1. Зобов'язання Уряду;
2. Політичні і правові засади;
3. Інституційна структура, розподіл відповідальності та спроможність урядових структур;
4. Політика та процедури Уряду стосовно обробки даних;
5. Попит на відкриті дані;
6. Залучення громадського сектору та можливості для відкритих даних;
7. Фінансування політики відкритих даних;
8. Національна технологічна інфраструктура та навички.

Якщо коротко, з готовністю громадськості використовувати відкриті дані все добре, більш-менш хороша ситуація з готовністю уряду відкривати дані, технічною інфраструктурою, але все погано в плані фінансування, індустрією розробки продуктів на базі відкритих даних і захистом приватності. Детальний звіт знаходиться за [посиланням](http://dhrp.org.ua/uk/publikatsii1/1071-20160227-ua-publication) (є звіти на українській і англійській мовах).

## Дорожня карта розвитку відкритих даних

З метою забезпечення комплексного розвитку відкритих даних Державним агентством з питань електронного урядування напрацьовано Дорожню карту розвитку відкритих даних в Україні, яка містить 41 завдання по 5 напрямкам:

1. Підвищення доступності та якості відкритих даних;
2. Розвиток спроможності органів влади щодо публікації відкритих даних;
3. Посилення ролі відкритих даних в реалізації державної політики;
4. Нормативно-правове забезпечення;
5. Розвиток попиту та спроможності цільових аудиторій щодо використання відкритих даних.

Зазначений документ затверджено наказом Мінрегіону від 04.02.2016 року № 19. Повний текст розміщено за [посиланням](https://drive.google.com/file/d/0B1kGsKt9XV_QaFZVaTZiT19aRTA/view.).

## Хартія відкритих даних

Наразі Україною ініціюється приєднання до міжнародної Хартії відкритих даних.

Проект Постанови КМУ знаходиться за [посиланням](https://drive.google.com/file/d/0B1kGsKt9XV_QMEFNNklzRXdaSDA/view?usp=sharing).

Розробка Міжнародної хартії була ініційована представниками урядів Канади, Мексики, Великобританії, впливових міжнародних організацій у травні 2015 року під час міжнародної конференції з питань відкритих даних у Канаді.

Головною метою Міжнародної Хартії відкритих даних є покращення та сприяння співпраці та взаємоузгодженості для прийняття та реалізації спільних принципів, стандартів та кращих практик відкритих даних по всьому світу. Цілями Хартії є поширення демократії, боротьба з корупцією та сприяння економічному зростанню по всьому світу. Хартія визначає 6 головних принципів та шляхи розвитку відкритих даних для країни.

## Світові та українські компетенції й рейтинги

Сьогодні найбільш важливими є наступні два рейтинги оцінювання стану розвитку відкритих даних:

1. [Open Data Barometer](http://opendatabarometer.org/data-explorer/?_year=2015&indicator=ODB&open=UKR). За цим рейтингом Україна посідає 62 місце.

2. [Open Data Index](http://index.okfn.org/place/ukraine). За цим рейтингом Україна посідає 54 місце (із 122).

Україна співпрацює з такими міжнародними організаціями в сфері відкритих даних: 

1. [Open Data Institute](http://theodi.org). Займається розвитком відкритих даних по всьому світу, формуванням стандартів та єдиних підходів, розвитком компетенцій.

2. [Open knowledge foundation](https://okfn.org). Займається підтримкою громадських інституцій щодо розвитку відкритих даних, а також розвитком відкритої платформи для побудови порталів відкритих даних CKAN.

3. [Open Data for Development](http://od4d.net). Займається підтримкою ініціатив з розвитку відкритих даних по всьому світу, а також організацію міжнародної співпраці.

В останньому рейтингу E-Government Development Index (EGDI) 2016 Україна зайняла 62 місце серед 193 країн, покращивши свою позицію на 25 пунктів.

## Професійні українські організації і ініціативи

В Україні запущений інкубатор відкритих даних 1991, який системно займається відбором проектів, їх інкубацією і пошуком інвестицій. На сьогодні уже було два набори в інкубаційну програму. 

На початку року розпочався EGAP Challenge – конкурс ІТ проектів в області електронної демократії, соціальної сфери і проектів в сфері відкритих даних. Спільна ініціатива Державного агентства з питань електронного урядування, Фонду Східної Європи в рамках реалізації Програми EGAP, що фінансується Швейцарською Конфедерацією, має на меті не тільки запровадити нові інструменти e-democracy в чотирьох регіонах України (Вінницький, Волинській, Дніпровській та Одеській областях), а також показати українським стартапам нову нішу, в якій можливо створювати якісні проекти, що мають змогу безпосередньо впливати на місцеву і центральну влади. А влада – у свою чергу – отримає набір нових інструментів для росту ефективності її роботи і прозорості взаємодії з платниками податків.

Пріоритетними є напрямки:

* створення нових інструментів взаємодії влади і суспільства, особливо в частині надання можливості громадянам напряму впливати на процеси прийняття управлінських рішень;

* підвищення прозорості і відкритості діяльності органів влади, особливо в частині формування і виконання бюджетів, надання дозвільних документів і тому подібне;

* створення нових якісних сервісів для громадян і бізнесу, особливо в частині надання публічних послуг;

* розвиток проектів на базі відкритих даних;

* вирішення соціальних проблем;

* об'єднання і налагодження ефективного співробітництва громадян для вирішення загальних проблем;

* проекти в області Smart City;

* галузеві проекти, спрямовані на підвищення ефективності державного управління і обслуговування громадян та бізнесу (е-екологія, е-медицина, е-освіта і т.д.).

В чотирьох областях пройшли креативні уікенди на яких було відібрано 15 проектів для проходження двомісячної інкубації. Партнерами конкурсу виступили компанії Cisco, IBM, DeNovo, Intel.

В Україні діють потужні громадські організації, що працюють з відритими даними— ОПОРА, Чесно, Канцелярська сотня, Vox Ukraine і інші. Результат — десятки досліджень, проекти по візуалізації даних, моніторинг відкритих наборів даних, десятки заходів по усій країні. Декілька проектів уже імплементовані в державні служби і проекти "смарт сіті". 

Наприклад, відкриті дані Укрзалізниці дали можливість провести масштабне дослідження – хто куди подорожує, які ключові станції, як розподіляється потік пасажирів і інші візуалізації.

## Рекомендації по роботі з відкритими даними

На сайті Верховної Ради є [інструкція](http://data.rada.gov.ua/open/main/opendata) по роботі з відкритими даними.

Також спільно з Агентством з питань електронного врядування та ПРООН були розроблені [методичні рекомендації щодо оприлюднення наборів даних у формі відкритих даних](https://drive.google.com/file/d/0B1kGsKt9XV_QWjhaZ0ZMVmFiUE0/view).

Метою рекомендацій є ознайомлення відповідальних осіб розпорядників інформації з ключовими питаннями, що постають у процесі оприлюднення наборів даних у формі відкритих даних. Також, методичні рекомендації будуть корисні розробникам застосунків на базі відкритих даних, громадським організаціям, засобам масової інформації та громадянам, які мають намір працювати з відкритими даними.

Проект першої версії Методичних рекомендацій підготовлено на основі найбільш популярних загальних запитань від розпорядників інформації, але, безумовно, ще не охоплює усіх піднятих розпорядниками інформації та громадськістю питань. Методичні рекомендації будуть систематично оновлюватися з метою забезпечення розпорядників інформації необхідною інформацією для оприлюднення наборів даних.

---

В цілому, Україна знаходиться лише на початку з точки зору відкритості державних служб, створення інститутів моніторингу діяльності депутатів, чиновників і міських служб, дигіталізації державних послуг. За наявності політичної волі, допомоги міжнародних організацій, а також за умови координації громадських організацій, бізнесу та ІТ сектору Україна здатна на досить короткий термін значно поліпшити свої показники в області електронного врядування та відкритих даних.
